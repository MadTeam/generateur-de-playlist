\documentclass{article}

\usepackage[utf8]{inputenc} %encodage en utf-8 pour la console UNIX
\usepackage[francais]{babel}
\usepackage{latex2man} %importation du package permettant de générer un document LaTeX en manpage

\begin{document}
intro :
le générateur de playlist permet de générer un fichier dans un format spécial qui contient un ensemble de titre musical selon les choix de l'utilisateur

contenu :
le packet contient les fichiers :
-main.py -> programme principal
-config.py -> fichier de configuration
-playlist.log -> fichier de log
[-docs/manual.pdf -> le présent manuel]
[-docs/manual.tex -> le fichier LaTeX du manuel]
[-docs/manual.manpage -> le fichier permettant une manpage]
[-docs/manual.html -> le manuel au format internet]

pré-requis :
python 3

utilisation :
./generator name format length [-h] [-v]
**************legende : les arguments entre crochets sont optionnels******************

arguments obligatoires :
-name : le nom du fichier de sortie
-format : permet de choisir le format du fichier de sortie
le format doit correspondre à l'un des 3 choix proposés {xspf,m3u,pls}
-length : la durée totale de la playlist en minutes
la durée choisie doit être supérieure à 0 et ne comporter aucune lettre

arguments optionnels :
-h --help : affiche l'aide (en anglais)
-v --verbose : permet l'affichage détaillé des opérations (ces informations sont également disponible dans le fichier de log)



\end{document}
